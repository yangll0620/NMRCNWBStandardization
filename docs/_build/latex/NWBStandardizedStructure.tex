%% Generated by Sphinx.
\def\sphinxdocclass{report}
\documentclass[letterpaper,10pt,english]{sphinxmanual}
\ifdefined\pdfpxdimen
   \let\sphinxpxdimen\pdfpxdimen\else\newdimen\sphinxpxdimen
\fi \sphinxpxdimen=.75bp\relax

\PassOptionsToPackage{warn}{textcomp}
\usepackage[utf8]{inputenc}
\ifdefined\DeclareUnicodeCharacter
% support both utf8 and utf8x syntaxes
  \ifdefined\DeclareUnicodeCharacterAsOptional
    \def\sphinxDUC#1{\DeclareUnicodeCharacter{"#1}}
  \else
    \let\sphinxDUC\DeclareUnicodeCharacter
  \fi
  \sphinxDUC{00A0}{\nobreakspace}
  \sphinxDUC{2500}{\sphinxunichar{2500}}
  \sphinxDUC{2502}{\sphinxunichar{2502}}
  \sphinxDUC{2514}{\sphinxunichar{2514}}
  \sphinxDUC{251C}{\sphinxunichar{251C}}
  \sphinxDUC{2572}{\textbackslash}
\fi
\usepackage{cmap}
\usepackage[T1]{fontenc}
\usepackage{amsmath,amssymb,amstext}
\usepackage{babel}



\usepackage{times}
\expandafter\ifx\csname T@LGR\endcsname\relax
\else
% LGR was declared as font encoding
  \substitutefont{LGR}{\rmdefault}{cmr}
  \substitutefont{LGR}{\sfdefault}{cmss}
  \substitutefont{LGR}{\ttdefault}{cmtt}
\fi
\expandafter\ifx\csname T@X2\endcsname\relax
  \expandafter\ifx\csname T@T2A\endcsname\relax
  \else
  % T2A was declared as font encoding
    \substitutefont{T2A}{\rmdefault}{cmr}
    \substitutefont{T2A}{\sfdefault}{cmss}
    \substitutefont{T2A}{\ttdefault}{cmtt}
  \fi
\else
% X2 was declared as font encoding
  \substitutefont{X2}{\rmdefault}{cmr}
  \substitutefont{X2}{\sfdefault}{cmss}
  \substitutefont{X2}{\ttdefault}{cmtt}
\fi


\usepackage[Bjarne]{fncychap}
\usepackage{sphinx}

\fvset{fontsize=\small}
\usepackage{geometry}

% Include hyperref last.
\usepackage{hyperref}
% Fix anchor placement for figures with captions.
\usepackage{hypcap}% it must be loaded after hyperref.
% Set up styles of URL: it should be placed after hyperref.
\urlstyle{same}

\usepackage{sphinxmessages}
\setcounter{tocdepth}{1}



\title{NMRC NWB Standardized Structure}
\date{May 16, 2019}
\release{1.0}
\author{Lingling Yang}
\newcommand{\sphinxlogo}{\vbox{}}
\renewcommand{\releasename}{Release}
\makeindex
\begin{document}

\pagestyle{empty}
\sphinxmaketitle
\pagestyle{plain}
\sphinxtableofcontents
\pagestyle{normal}
\phantomsection\label{\detokenize{index::doc}}


Contents:


\chapter{About NWB}
\label{\detokenize{aboutNWB:about-nwb}}\label{\detokenize{aboutNWB::doc}}
Neurodata Without Borders (NWB) format is unified data format designed in the Neurodata Without Borders: Neurophysiology (NWB:N) to incorporate several types of data, including electrophysiological and optical physiology data, stimuli and behavior data et.al.

The official website of NWB:N project can be found \sphinxhref{https://neurodatawithoutborders.github.io/}{here}.


\chapter{Pre-installed Toolboxes}
\label{\detokenize{pretoolboxes:pre-installed-toolboxes}}\label{\detokenize{pretoolboxes::doc}}
\begin{sphinxShadowBox}
\sphinxstyletopictitle{Contents}
\begin{itemize}
\item {} 
\phantomsection\label{\detokenize{pretoolboxes:id3}}{\hyperref[\detokenize{pretoolboxes:pre-installed-toolboxes}]{\sphinxcrossref{Pre-installed Toolboxes}}}
\begin{itemize}
\item {} 
\phantomsection\label{\detokenize{pretoolboxes:id4}}{\hyperref[\detokenize{pretoolboxes:matnwb}]{\sphinxcrossref{MatNWB}}}

\item {} 
\phantomsection\label{\detokenize{pretoolboxes:id5}}{\hyperref[\detokenize{pretoolboxes:tdtmatlabsdk}]{\sphinxcrossref{TDTMatlabSDK}}}

\end{itemize}

\end{itemize}
\end{sphinxShadowBox}

\begin{sphinxadmonition}{important}{Important:}
Please installed the following toolboxes before using NWB standardized processing codes.
\end{sphinxadmonition}


\section{MatNWB}
\label{\detokenize{pretoolboxes:matnwb}}\label{\detokenize{pretoolboxes:installmatnwb-label}}\begin{quote}

MatNWB is the Matlab interface for reading and writing NWB file. To generate and use NWB structure, MatNWB should be inside the folder /util/.
\begin{enumerate}
\def\theenumi{\arabic{enumi}}
\def\labelenumi{\theenumi .}
\makeatletter\def\p@enumii{\p@enumi \theenumi .}\makeatother
\item {} 
Download the \sphinxhref{https://github.com/NeurodataWithoutBorders/matnwb}{MatNWB} .
\begin{quote}
\end{quote}

\item {} 
From the Matlab command line, generate matlab m-files:

\begin{sphinxVerbatim}[commandchars=\\\{\}]
\PYG{n}{generateCore}\PYG{p}{(}\PYG{l+s+s1}{\PYGZsq{}}\PYG{l+s+s1}{schema/core/nwb.namespace.yaml}\PYG{l+s+s1}{\PYGZsq{}}\PYG{p}{)}\PYG{p}{;}
\end{sphinxVerbatim}

\item {} 
Copy the folder matnwb into /util/

\end{enumerate}
\end{quote}


\section{TDTMatlabSDK}
\label{\detokenize{pretoolboxes:tdtmatlabsdk}}\label{\detokenize{pretoolboxes:installtdtmatsdk-label}}\begin{quote}

TDTMatlabSDK is the Matlab TDT data import tool. TDTMatlabSDK should be inside the folder /util/ when converting tdt data to NWB structure.
\begin{enumerate}
\def\theenumi{\arabic{enumi}}
\def\labelenumi{\theenumi .}
\makeatletter\def\p@enumii{\p@enumi \theenumi .}\makeatother
\item {} 
Download the zipped \sphinxhref{https://www.tdt.com/support/examples/TDTMatlabSDK.zip}{TDTMatlabSDK} tool.
\begin{quote}
\end{quote}

\item {} 
Extrac the zip files into /util/ folder

\end{enumerate}
\end{quote}


\chapter{Input Systems}
\label{\detokenize{inputsystems:input-systems}}\label{\detokenize{inputsystems::doc}}

\section{TDT System}
\label{\detokenize{systemtdt:tdt-system}}\label{\detokenize{systemtdt::doc}}

\subsection{Structure of TDT System}
\label{\detokenize{systemtdt:structure-of-tdt-system}}

\begin{savenotes}\sphinxattablestart
\centering
\begin{tabular}[t]{|*{6}{\X{1}{6}|}}
\hline
\sphinxstyletheadfamily 
tdt filed
&\sphinxstyletheadfamily 
sub-field
&\sphinxstyletheadfamily 
chair
&\sphinxstyletheadfamily 
gait
&\sphinxstyletheadfamily 
description
&\sphinxstyletheadfamily 
user def.
\\
\hline\sphinxmultirow{3}{7}{%
\begin{varwidth}[t]{\sphinxcolwidth{1}{6}}
.epocs
\par
\vskip-\baselineskip\vbox{\hbox{\strut}}\end{varwidth}%
}%
&
.Cam1
&
yes
&
yes
&\sphinxmultirow{2}{11}{%
\begin{varwidth}[t]{\sphinxcolwidth{1}{6}}

\begin{DUlineblock}{0em}
\item[] Cam1/2:  onset and offset time
\item[] of each frame
\end{DUlineblock}
\par
\vskip-\baselineskip\vbox{\hbox{\strut}}\end{varwidth}%
}%
&
n/a
\\
\cline{2-4}\cline{6-6}\sphinxtablestrut{7}&
.Cam2
&
no
&
yes
&\sphinxtablestrut{11}&
n/a
\\
\cline{2-6}\sphinxtablestrut{7}&
.Spd\_
&
no
&
yes
&
\begin{DUlineblock}{0em}
\item[] {\color{red}\bfseries{}Spd\_}: onset  and  offset  time
\item[] of gait mat
\end{DUlineblock}
&
Spdg
\\
\hline\sphinxmultirow{5}{22}{%
\begin{varwidth}[t]{\sphinxcolwidth{1}{6}}
.steams
\par
\vskip-\baselineskip\vbox{\hbox{\strut}}\end{varwidth}%
}%
&
.BUGG
&
yes
&
yes
&
BUGG: store the neural data

{[}n\_chns,  n\_temporal{]}

start\_time = 0
&
Neur
\\
\cline{2-6}\sphinxtablestrut{22}&
.EYEa
&
yes
&
no
&
x,y positions of eyes

{[}2, n\_temporal{]}

start\_time = 9.5367e-07
&
EYEa
\\
\cline{2-6}\sphinxtablestrut{22}&
.EYEt
&
yes
&
no
&
sync data from eye tracking system

{[}1, n\_temporal{]}

start\_time = 9.5367e-07
&
EYEt
\\
\cline{2-6}\sphinxtablestrut{22}&
.Stpd
&
yes
&
no
&
\begin{DUlineblock}{0em}
\item[] Stpd: synchronization signal
\item[] from the touch pad.
\end{DUlineblock}

{[}1, n\_temporal{]}

start\_time = 9.5367e-07
&
Stpd
\\
\cline{2-6}\sphinxtablestrut{22}&
.Para
&
yes
&
no
&
For what?

{[}4,  n\_temporal{]}

start\_time = 9.5367e-07
&
Para
\\
\hline
\end{tabular}
\par
\sphinxattableend\end{savenotes}

Example dataset:
\begin{itemize}
\item {} 
setup-chair: Bug-190111 -\textgreater{} Block-1

\item {} 
setup-gait: Bug-181130 -\textgreater{} Block-1

\end{itemize}


\subsection{NWB Structure Storing TDT data}
\label{\detokenize{systemtdt:nwb-structure-storing-tdt-data}}

\section{Motion Analysis System}
\label{\detokenize{systemma:motion-analysis-system}}\label{\detokenize{systemma::doc}}

\section{Eye Tracking System}
\label{\detokenize{systemeyetracking:eye-tracking-system}}\label{\detokenize{systemeyetracking::doc}}

\section{Gait Mat System}
\label{\detokenize{systemgaitmat:gait-mat-system}}\label{\detokenize{systemgaitmat::doc}}

\chapter{Functions}
\label{\detokenize{functions:functions}}\label{\detokenize{functions::doc}}

\section{Convert to NWB Functions}
\label{\detokenize{function_convert:convert-to-nwb-functions}}\label{\detokenize{function_convert::doc}}

\section{Read from NWB Structure}
\label{\detokenize{function_read:read-from-nwb-structure}}\label{\detokenize{function_read::doc}}


\renewcommand{\indexname}{Index}
\printindex
\end{document}